\section{Heuristics in the Adaptive A*}

\paragraph{Problem}
The project argues that "the Manhattan distances are consistent in gridworlds
in which the agent can move only in the four main compass directions." Prove
that this is indeed the case.

\paragraph{Solution}
Assume that Manhattan distances are not consistent in gridworlds, then there
must exist some cell $n$, which $h(n) > h(n') + c(n,n')$, where $n'$ is the
next cell generated by $n$ and $c$ is the cost from $n$ to $n'$. In the
gridworlds where the agent can move only in the four main compass directions,
the $c(n,n')$ is always 1. Then in order to make the inequality above hold
true, difference of $h(n)$ and $h(n')$ must be larger than 1. However, if the
agent can move only in the four main compass directions, the cell $n$ and $n'$
must be adjacent, the different of their heuristics values must be 1. It is
conflict. So the Manhattan distances are consistent in this case.

\paragraph{Problem}
Furthermore, it is argued that "The h-value $h_{new}(s)$... are only admissible
but also consistent." Prove that the Adaptive A* leaves initially consistent
h-value consistent even if action costs can increase.

\paragraph{Solution}
By referring to \cite{koenig2006real}, we conclude that for every cell $n$ and
every successor $n'$ of $n$, when the h value update, there are 3 cases.

First, if both $n$ and $n'$ were expanded, it means that $h_{new}(n) = g(goal) -
g(n)$ and $h_{new}(n') = g(goal) - g(n')$. Then $h_{new}(n) - h_{new}(n') =
g(goal) - g(n) - g(goal) + g(n')=g(n') - g(n) \leq c$, because $n$ and $n'$ are
adjacent. So $h_{new}(n) \leq h_{new}(n') + c$. 

Second, $n$ was expanded but $n'$ was not. Then we have $h_{new}(n) = g(goal) -
g(n)$, $g(n') \leq g(n) + c $, according to case 1, $f(goal) \leq f(n')$, since
$n'$ was generated but not expanded. Thus,

\begin{equation*}
  \begin{aligned}
   h_{new}(n) &= g(goal) - g(n)\\
   &= f(goal) - g(n)\\
   &\leq f(n') -g(n)\\
   &= g(n') + h(n') -g(n)\\
   &= g(n') + h_{new}(n') -g(n)\\
   &\leq g(n') + h_{new}(n') - g(n') + c\\
   &= h_{new}(h) + c
  \end{aligned}
\end{equation*}

Third, $n$ was not expanded,which implies that $h_{new}(n) = h(n)$. In the
meantime, $h(n') \leq h_{new}(n')$. Thus $h_{new}(n) = h(n) \leq h(n') + c \leq
h_{new}(n') + c$.

Therefore, the Adaptive A* leaves initially consistent h-value consistent even
if action costs can increase.

