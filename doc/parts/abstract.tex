\section*{Abstract}

Heuristic search algorithms like A* can be adapted to solving path planning
problems of directed goal and unknown environment. In this report, we mainly
discuss and evaluate three variants of A* algorithms, namely Repeated Forward
A*, Repeated Backward A* and Adaptive A*. In the experiment, we first generate
two sets of 50 random grid-based maps, where one follows the assignment
requirement containing approximate 30\% obstacles and 70\% roads, and the other
is the set of corridor-like mazes generated by DFS with randomly expanded
nodes.  Then we compare three algorithms by such evaluation indexes as number
of expanded nodes, explored nodes, moves, optimal moves, etc. The result shows
that 1) original Repeated Forward A* expands states 30 times more than the
modified one with $c\times f-g$; 2) Repeated Backward A* expands 15 times more
states than Repeated Forward A*; 3) Adaptive A* expands 22\% less states than
Repeated Forward A* only in complicated mazes. Finally, we discuss and calculate
how to optimize data structures to store states as many as possible within only
4M memory. This is the practical problem in the situation where computational
resources are rare and precious.
