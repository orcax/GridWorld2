\section*{Abstract}

Heuristic search algorithms like A* can be adapted to solving path planning
problems of directed goal and unknown environment. In this report, we mainly
discuss and evaluate three variants of A* algorithms, namely Repeated Forward
A*, Repeated Backward A* and Adaptive A*. In the experiment, we first generate
two sets of 1000 random grid-based maps, where one follows the assignment
requirement containing approximate 30\% obstacles and 70\% roads, and the other
is the set of corridor-like mazes generated by DFS with randomly expanded
nodes.  Then we compare three algorithms by such evaluation indices as number
of expanded nodes, number of explored nodes, total cost of steps, optimal cost
of steps, etc. The result shows that 1) the average number of expanded nodes
per map and explored nodes per map of Adaptive A* are respectively 28.06\% and
24.00\% less than those of Repeated Forward A*; 2) cost of steps for all three
algorithms is pretty much the same. This indicates that Adaptive A* is greatly
optimized in path replanning phase, while there is no significant improvement
in actual moving phase. Finally, we discuss and calculate how to optimize data
structures to store states as many as possible within only 4M memory.  This is
the practical problem in the situation where computational resources are rare
and precious.
