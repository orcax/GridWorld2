\section{Memory Issues}

\paragraph{Problem}
You performed all experiments in gridworlds of size $101\times 101$ but some
real-time computer games use maps whose number of cells is up to two orders of
magnitude larger than that. It is then especially important to limit the amount
of information that is stored per cell. For example, the tree-pointers can be
implemented with only two bits per cell. Suggest additional ways to reduce the
memory consumption of your implementations further. Then, calculate the amount
of memory that they need to operate on gridworlds of size $1001\times 1001$ and
the largest gridworld that they can operate on within a memory limit of 4
MBytes.

\paragraph{Solution}
The data structure of states in our project show below.

\begin{tabular}{|l|l|l|l|}
\hline
\multicolumn{4}{|c|}{State}\\
\hline
type&name&bits&description\\
\hline
\multirow{2}{*}{Integer}
&column&10&the column of state in grid\\
&row&10&the row of state in grid\\
\hline
Boolean&bSearch&1&reflect whether the cell of this state have been found\\
\hline
Boolean&canMove&1&reflect whether the cell of this state is viable\\
\hline
\end{tabular}\\

Let's denote $n$ as the memory size of each state and $N$ as the memory size of
the array of states. In this data structure, $n = 22$ bits. In the map of size
$1001 \times 1001$, then

\begin{equation*}
  N = n*1001*1001 = 2.6 MBytes.
\end{equation*}

Denote $g(s)$ as the cost from start state to state $s$. During each iteration,
namely the procedure of computing path, the array of $g$ values need store all
cells which have been explored. 

Denote $h(s)$ as the estimated cost from state $s$ to goal state. During each
iteration, namely the procedure of computing path, the array of $h$ values need
store all cells which have been explored. 

Denote $search(s)$ as the explore times of state $s$. During each iteration,
the array of $search$ values need store all cells which have been explored. 

Denote $close(s)$ as the state which have been expanded. During each iteration,
the array of $close$ values need store all states which have been expanded.

Denote $open(s)$ as the priority heap that stores states $s$ to be expand
sorted by the sum of heuristic value and $g(s)$. The heap $open$ need store
nearly all cells which have been explored. 

The type of the value of each element in the data structure above is integer.

Assume the start state and goal state are in the opposite endpoints in the
diagonal line of the map. Then for each iteration, the number of expanded cells
will be close to the sum of $rows$ and $cols$, where $rows$ is the total number
of rows in map and $cols$ is the total number of columns. So the number of
expanded cells $\approx$ 2000. Then in the map of size $1001 \times 1001$, 

\begin{equation*}
  \begin{aligned}
   &g = 4 * 2000 * (32 + 34) \approx 65 KByte\\
   &h = 4 * 2000 * (32 + 34) \approx 65 KByte\\
   &search = 4 * 2000 * (32 + 34) \approx 65 KByte\\
   &close = 2000 * 34 \approx 8 KByte\\
   &open = 4 * 2000 * 34 \approx 65 KByte\\
   &tree = 4 * 2000 * (32 + 34) \approx 65 KByte\\
  \end{aligned}
\end{equation*}

So, the total memory cost is : 

$$S_{total} = S +  g + search + close + open + tree \approx 3 MBytes < 4MBytes$$

Then the project can operate within a memory limit of 4 Mbyte, using a map of
size 1001 $\times$ 1001.
