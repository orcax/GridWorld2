\section{Memory Issues}

\paragraph{Problem}
You performed all experiments in gridworlds of size $101\times 101$ but some
real-time computer games use maps whose number of cells is up to two orders of
magnitude larger than that. It is then especially important to limit the amount
of information that is stored per cell. For example, the tree-pointers can be
implemented with only two bits per cell. Suggest additional ways to reduce the
memory consumption of your implementations further. Then, calculate the amount
of memory that they need to operate on gridworlds of size $1001\times 1001$ and
the largest gridworld that they can operate on within a memory limit of 4
MBytes.

\paragraph{Solution}
The data structure of states in our project show below.

\begin{tabular}{|l|l|l|l|}
\hline
\multicolumn{4}{|c|}{state}\\
\hline
type&name&bits&description\\
\hline
\multirow{2}{*}{Integer}
&column&10&the column of state in grid\\
&row&10&the row of state in grid\\
\hline
Boolean&bSearch&1&reflect whether the cell of this state have been found\\
\hline
Boolean&canMove&1&reflect whether the cell of this state is viable\\
\hline
\end{tabular}\\

Denote s as the the space of each state, S as the space of the array of states.Using this data structure, s = 22 bits. In the map of size 1001 $\times$ 1001, then
\begin{equation*}
  \begin{aligned}
   &S = s * 1001 * 1001 = 2.6 Mbyte.
  \end{aligned}
\end{equation*}

Denote g(s) as the cost from start state to state s, during each iteration, namely the procedure of compute path, the array g need to store all notes which have been explored. 
Denote h(s) as the estimate cost from state s to goal state, during each iteration, namely the procedure of compute path, the array h need to store all notes which have been explored. 
Denote search(s) as the explore times of state s, during each iteration,the array s need to store all notes which have been explored. 
Denote close(s) as the state which have been expanded,during each iteration, the array cost close need to store all states which have been expand.
Denote open(s) as the state s will be expand in a priority heap sorted by the sum of heuristic value and g(s). The heap open need to store nearly all notes which have been explored. 
The type of the value of each elements in the data structure above is integer.
Assume the start state and goal state are in the opposite endpoint in the 
diagonal line of the map. Then each iteration, the number of expand will near the sum of rows and columns, which rows is the total number of rows in map and columns is the total number of columns. So the number of expand $\approx$ 2000.
Then 
$\times$ 1001, then
\begin{equation*}
  \begin{aligned}
   &g = 4 * 2000 * (32 + 34) \approx 65 KByte\\
   &h = 4 * 2000 * (32 + 34) \approx 65 KByte\\
   &search = 4 * 2000 * (32 + 34) \approx 65 KByte\\
   &close = 2000 * 34 \approx 8 KByte\\
   &open = 4 * 2000 * 34 \approx 65 KByte\\
   &tree = 4 * 2000 * (32 + 34) \approx 65 KByte\\
  \end{aligned}
\end{equation*}
so, the total space used: 
$S_{total} = S +  g + search + close + open + tree \approx 3 MByte$, which is less than 4 Mbyte.Then the project can operate within a memory limit of 4 Mbyte, using a map of size 1001 $\times$ 1001.\\
