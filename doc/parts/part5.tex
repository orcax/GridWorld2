\section{Heuristics in the Adaptive A*}

\paragraph{Problem}
Implement and compare Repeated Forward A* and Adaptive A* with respect to their
runtime. Explain your observations in detail, that is, explain what you
observed and give a reason for the observation. Both search algorithms should
break ties among cells with the same f-value in favor of cells with larger
g-values and remaining ties in an identical way, for example randomly.

\paragraph{Solution}
Firstly, we use 50 maps of size $101\times 101$ with discrete obstacles to
evaluate performance of two algorithms. The result is shown in Table
\ref{tbl:rpa-rtaa} and complete experimental data for RTAA* is listed in
Appendix C. As we can see, the average number of cells expanded per map in RFA*
is almost equivalent to that in RTAA* (Real-Time Adaptive A*).  Moreover, rest
of evaluation indexes also reflect no significant difference between these
algorithms. Therefore, can we draw the conclusion that the running time of
RTAA* is approximately equal to RFA*?

\begin{table}[ht]
\centering
\caption{Result of Repeated Forward A* and Adaptive A* in Maps}
\begin{tabular}{|l|l|l|l|l|}
\hline
Algorithm & Aver. Expa/M & aver. Expa/P & Aver. Expl/M & Aver. Expl/P \\
\hline
RFA* & 8847.88 & 92.44 & 25461.42 & 267.68 \\
\hline
RTAA* & 8749.2 & 91.32 & 25413.66 & 266.72 \\
\hhline{|=|=|=|=|=|}
Algorithm & Aver. Moves & Aver. Count & Aver. Optimal & Aver. Ratio \\
\hline
RFA* & 335.72 & 88.2 & 198.6 & 1.6852 \\
\hline
RTAA* & 367 & 96.18 & 198.6 & 1.843 \\
\hline
\end{tabular}
\label{tbl:rpa-rtaa}
\end{table}

THE ANSWER IS NO!!!

And this is why we generate another kind of gridworld, namely the maze. As
previously said, any maze in this case is always a connected undirected graph,
so the optimal path from start to goal is quite long, full of twists and turns.
It is this characteristic of maze that favors the RTAA* to performs better than
RFA*. The result is shown in Table \ref{tbl:rpa-rtaa2}, and complete
experimental data is listed in Appendix C. The average number of expanded cells
in each map for RTAA* is about 22.3\% less than that for RFA*, and the average
number of expanded cells in each planning for RTAA* is about 19.09\% less than
that for RFA*. Meanwhile, the average number of explored cells for RTAA* is
almost the same as that for RFA* because both algorithms will place all
successors into $OPEN$ list. The average number of moves either in each map or
in each planning for RTAA* is 9.3\% larger than that for RFA*. Compared this
ratio to that of expanded cells, we can find RTAA* performs better in running
time of expansion by the price of moving more steps.

\begin{table}[ht]
\centering
\caption{Result of Repeated Forward A* and Adaptive A* in Mazes}
\begin{tabular}{|l|l|l|l|l|}
\hline
Algorithm & Aver. Expa/M & aver. Expa/P & Aver. Expl/M & Aver. Expl/P \\
\hline
RFA* & 148454.72 & 148.32 & 338600.96 & 349.84 \\
\hline
RTAA* & 115318.38 & 120 & 301468.68 & 319.16 \\
\hhline{|=|=|=|=|=|}
Algorithm & Aver. Moves & Aver. Count & Aver. Optimal & Aver. Ratio \\
\hline
RFA* & 2848.6 & 932.54 & 1078.48 & 2.87 \\
\hline
RTAA* & 2854.56 & 929.44 & 1078.48 & 2.8758 \\
\hline
\end{tabular}
\label{tbl:rpa-rtaa2}
\end{table}

The advantage of RTAA* lies in its constant update for heuristic value of
expanded states. The heuristics of the same state in different planning
procedures are monotonically nondecreasing over time, and thus become more
informed\cite{koenig2006real}. Since in every iteration, the algorithm will
choose the cell with the smallest $f$-value in the $OPEN$ list, the cell with
updated larger heuristic, namely larger $f$-value, implies that it will be less
probable to be extracted from the $OPEN$ list. Instead, some other cells may be
firstly selected by RTAA*, which is selected later by RFA*. On the other hand,
the heuristic value in RTAA* can be regarded composed of two parts. The first
part is calculated according to from gained knowledge in previous planning, and
the second the part is the ordinary estimate goal distance, like Manhattan
distance in the case. The first part can benefit RTAA* from selecting a more
informed cell from $OPEN$ list, so that less nodes are needed to be expanded,
whereas RFA* expands more redundant cells during planning.

