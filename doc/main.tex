\documentclass[12pt]{article}
\usepackage[a4paper]{geometry}
\usepackage{graphicx}
\usepackage{amsmath}
\usepackage{amsfonts}
\usepackage{titlesec}
\usepackage{hyperref}

\title{%
\textbf{$\S$CS 520: Assignment 1 } \\
Fast Trajectory Replanning
\vspace{15em}}

\author{%
Xiaoyang Xie \\ 167008240 
\and
Yikun Xian \\ 168000142}

\date{%
Department of Computer Science \\
Rutgers University, New Brunswick, NJ \\
\vspace{7em}
09 Octorber 2015}
\begin{document}

\maketitle\thispagestyle{empty}
\clearpage

\setcounter{page}{1}
\setlength\parindent{0pt}
\graphicspath{figures/}

\titleformat{\section}{\Large\bfseries}{\thesection}{1em}{}
\section*{Abstract}
Heuristic search algorithms like A* can be adapted to solving path planning 
problems of directed goal and unknown environment In this report, we mainly 
discuss and evaluate three variants of A* algorithms, namely Repeated Forward 
A*, Repeated Backward A* and Adaptive A*. In the experiment, we first generate 
two sets of 1000 random grid-based maps, where one follows the assignment 
requirement containing approximate 30\% obstacles and 70\% roads, and the other 
is the set of corridor-like mazes generated by DFS with randomly expanded nodes. 
Then we compare three algorithms by such evaluation indices as number of 
expanded nodes, number of explored nodes, total cost of steps, optimal cost of 
steps, etc. The result shows that 1) the average number of expanded nodes per 
map and explored nodes per map of Adaptive A* are respectively 28.06\% and 
24.00\% less than those of Repeated 
Forward A*; 2) cost of steps for all three algorithms is pretty much the same. 
This indicates that Adaptive A* is greatly optimized in path replanning phase, 
while there is no significant improvement in actual moving phase. Finally, we 
discuss and calculate how to optimize data structures to store states as many 
as possible within only 4M memory. This is the practical problem in the situation
where computational resources are rare and precious.

\titleformat{\section}{\Large\bfseries}{Part \thesection}{1em}{}
\setcounter{section}{-1}
\section{Setup Environment}

We simulate all path finding processes based on the framework of 
GridWorld\cite{web:gridworld}, an AP case study project from collegeboard
\footnote{https://www.collegeboard.org/}. It provides graphical user interface 
based on Java AWT where visual objects can interact and perform customized 
actions in a two-dimensional grid map. In the next part, we will first illustrate
original GridWorld framework and our enhancement of displaying colored path.
This mainly involves the engineering work, so if you want to directly delve 
into algorithm analysis, please skip it.

\subsection{GridWorld Architecture and Modification}

The framework structure of original GridWorld project are divided into four parts: 
Actor, Grid, GUI and World, as shown in Figure \ref{fig:framework-structure}.
The Actor package contains objects whose behavior on the map can be arbitrarily 
defined by rewriting following method in each inherited class:

\begin{lstlisting}
%%@Override%%
public void act()
\end{lstlisting}

\begin{figure}[ht]
\centering
\includegraphics[width=0.5\textwidth]{framework-structure}
\caption{GridWorld package structure}
\label{fig:framework-structure}
\end{figure}

The Grid package defines bounded grid
map features and connection to actors. The GUI package encapsulates low-level
Java AWT to provide APIs for visualization and interactions of map and actors.
The World package provides high-level integration of actors and world.

In order to visualize the presumed unblocked path and the set of nodes expanded
and explored after each planning, we enhance the $GridPanel$ class in the 
GUI package by implementing following method:

\begin{lstlisting}
private void drawColoredLocations(Graphics2D g2)
\end{lstlisting}

Meanwhile, following five abstract methods are added to $Grid$ class so that
inherited classes like $BoundedGrid$ and $UnboundedGrid$ should provide 
implementation of how to configure colors on each grid.
\begin{lstlisting}
ArrayList<Location> getColoredLocations();
Color getColor(Location loc);
void putColor(Location loc, Color color);
void removeColor(Location loc);
void resetColors();
\end{lstlisting}


\subsection{Maze Generation Algorithm}

\subsection{How to Run}

Our project is 

\section*{Part 1 - Understanding the Methods}

\section{The Effects of Ties}

\paragraph{Problem} 
Repeated Forward A* needs to break ties to decide which
cell to expand next if several cells have the same smallest f-value. It can
either break ties in favor of cells with smaller g-values or in favor of cells
with larger g-values. Implement and compare both versions of Repeated Forward
A* with respect to their runtime or, equivalently, number of expanded cells.
Explain your observations in detail, that is, explain what you observed and
give a reason for the observation.

\paragraph{Solution}

\section*{Part 3 - Forward vs. Backward}

\section*{Part 4 - Heuristics in the Adaptive A*}

\section*{Part 5 - Heuristics in the Adaptive A*}

\section*{Part 6 - Memory Issues}


\bibliographystyle{plain}
\bibliography{main}
\nocite{*}

\end{document}
