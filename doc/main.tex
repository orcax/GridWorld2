\documentclass[12pt]{article}
\usepackage[a4paper]{geometry}
\usepackage{graphicx}
\usepackage{amsmath}
\usepackage{amsfonts}
\usepackage{titlesec}
\usepackage{hyperref}

\title{%
\textbf{$\S$CS 520: Assignment 1 } \\
Fast Trajectory Replanning
\vspace{15em}}

\author{%
Xiaoyang Xie \\ 167008240 
\and
Yikun Xian \\ 168000142}

\date{%
Department of Computer Science \\
Rutgers University, New Brunswick, NJ \\
\vspace{7em}
09 Octorber 2015}
\begin{document}

\maketitle\thispagestyle{empty}
\clearpage

\setcounter{page}{1}
\setlength\parindent{0pt}
\graphicspath{figures/}

\titleformat{\section}{\Large\bfseries}{\thesection}{1em}{}
\section*{Abstract}
Heuristic search algorithms like A* can be adapted to solving path planning 
problems of directed goal and unknown environment In this report, we mainly 
discuss and evaluate three variants of A* algorithms, namely Repeated Forward 
A*, Repeated Backward A* and Adaptive A*. In the experiment, we first generate 
two sets of 1000 random grid-based maps, where one follows the assignment 
requirement containing approximate 30\% obstacles and 70\% roads, and the other 
is the set of corridor-like mazes generated by DFS with randomly expanded nodes. 
Then we compare three algorithms by such evaluation indices as number of 
expanded nodes, number of explored nodes, total cost of steps, optimal cost of 
steps, etc. The result shows that 1) the average number of expanded nodes per 
map and explored nodes per map of Adaptive A* are respectively 28.06\% and 
24.00\% less than those of Repeated 
Forward A*; 2) cost of steps for all three algorithms is pretty much the same. 
This indicates that Adaptive A* is greatly optimized in path replanning phase, 
while there is no significant improvement in actual moving phase. Finally, we 
discuss and calculate how to optimize data structures to store states as many 
as possible within only 4M memory. This is the practical problem in the situation
where computational resources are rare and precious.

\titleformat{\section}{\Large\bfseries}{Part \thesection}{1em}{}
\setcounter{section}{-1}
\section{Setup Environment}

We simulate all path finding processes based on the framework of 
GridWorld\cite{web:gridworld}, an AP case study project from collegeboard
\footnote{https://www.collegeboard.org/}. It provides graphical user interface 
based on Java AWT where visual objects can interact and perform customized 
actions in a two-dimensional grid map.

\subsection{GridWorld Architecture and Modification}

\subsection{Maze Generation Algorithm}

\subsection{How to Run}
Our project is 

\section{Understanding the Methods}

\section{The Effects of Ties}

\section*{Part 3 - Forward vs. Backward}

\section{Heuristics in the Adaptive A*}

\paragraph{Problem}
The project argues that “the Manhattan distances are consistent in gridworlds
in which the agent can move only in the four main compass directions.” Prove
that this is indeed the case.

\paragraph{Solution}

\paragraph{Problem}
Furthermore, it is argued that “The h-values hnew(s) ... are not only
admissible but also consistent.” Prove that Adaptive A* leaves initially
consistent h-values consistent even if action costs can increase.

\paragraph{Solution}

\section{Heuristics in the Adaptive A*}

\section*{Part 6 - Memory Issues}


\bibliographystyle{plain}
\bibliography{main}
\nocite{*}

\end{document}
